\subsection{Requirements Analysis}
In the remainder of this section, we detail the system's layout, regarding hardware and software, as well as unconsidered system characteristics. We proceed with an in-depth analysis of the problem and the requirements for solving it.
\\
The Huxley Messaging Hub system requirements are divided into functional and non-functional categories to ensure comprehensive coverage of system capabilities and constraints.

%% Market Analysis 
% Signal 
% matrix.org 

% Why choose our product

\subsubsection{Functional Requirements}
\begin{itemize}
    \item \textbf{User Management}: Support user registration and authentication in a local database
    \item \textbf{Message Handling}: Enable sending, receiving, and storing messages between registered users
    \item \textbf{Encryption}: Provide end-to-end encryption for all message communications
    \item \textbf{Status Indication}: Showcase visual feedback through LED indicators for system status
\end{itemize}

\subsubsection{Non-Functional Requirements}
\begin{itemize}
    \item \textbf{Security}: Ensure data security through message encryption
    \item \textbf{Performance}: Support multiple concurrent users with minimal latency
    \item \textbf{Feature Scalability}: Design for easy expansion of user base and features
    \item \textbf{User friendliness}: Create a user-friendly client interface
\end{itemize}

\subsection{System Constraints}
The constraints of Huxley are divided into technical and non-technical; the former provide us constraints that must be met, while the latter influence how we schedule and implement the project.

\subsubsection{Technical Constraints}
\begin{itemize}
    \item \textbf{Platform}: Must operate on Raspberry Pi 4 with limited processing power and memory
    \item \textbf{Programming Language}: Primary implementation in C++
    \item \textbf{Image}: Use a Linux kernel image generated by Buildroot.
    \item \textbf{Concurrency}: Use PThreads for concurrent tasks
    \item \textbf{Device Driver}: Implement at least one device driver
\end{itemize}

\subsubsection{Non-Technical Constraints}
\begin{itemize}
    \item \textbf{Small group}: Two people 
    \item \textbf{Budget}: Limited for additional hardware components
    \item \textbf{Timeline}: Semester-long development
\end{itemize}

\subsection{System Architecture}

% System Overview 
With the identified requirements and constraints in place, a high-level system overview was created. The core component is the Control Box, which holds the Raspberry Pi. Users will connect to the server through a mobile phone application GUI. The Raspberry hosts a lightweight SQLite database, and the LED blinks according \textcolor{red}{to the current System State (see below)}. % implement this FSM later


\begin{figure}[H]
    \centering
    \includegraphics[width=1\linewidth]{images/System Overview.drawio.png}
    \caption{System Overview}
    \label{fig:placeholder}
\end{figure}


% More design-ish
% All client traffic terminates at the \textbf{HuxleyServer} running on the Raspberry Pi within a local network (LAN). Clients connect over TCP; the Internet is \emph{not} required. The server coordinates authentication (AuthManager\,$\rightarrow$\,SQLite), message routing (MessageRouter\,$\rightarrow$\,CryptoEngine), persistence (SQLite), and hardware status (StatusManager\,$\rightarrow$\,LED driver).

% \begin{figure}[H]
%     \centering
%     \includegraphics[width=.4\linewidth]{images/system_architecture.png}
%     \caption{System Overview}
%     \label{fig:placeholder}
% \end{figure}

%TODO: System Overview graph here <----------





\section{Project Timeline}

The project's development is divided into numerous phases, each with its respective deadline.

% GANTT DIAGRAM
\begin{figure}[H]
    \centering
    \includegraphics[width=1\linewidth]{images/Gantt_Diagram.png}
    \caption{Gantt Diagram}
    \label{fig:Gantt Diagram}
\end{figure}